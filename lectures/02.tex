% 03.11.2023

\begin{exmpl}[Потенциальная функция для олигаоплии Курно на 3х игроках]
	\begin{align*}
		u_i(s) = (p - \sum_{j \in N}s_j) s_i - c s_i = (\underbrace{p - c}_{a} - \sum_{j \in N} s_j) s_i = a s_i - s_i \sum_{j \in N} s_j
	\end{align*}

	Тогда при $N = \{1, 2, 3\}$ :

	\begin{align*}
		u_1(s) &= as_1 - s_1(s_1 + s_2 + s_3) \\
		u_2(s) &= as_2 - s_2(s_1 + s_2 + s_3) \\
		u_3(s) &= as_3 - s_3(s_1 + s_2 + s_3)
	\end{align*}

	В потенциальный играх есть такое свойство ~--- пусть $u_i(s) = h_i(s) + g_i(s)$, тогда если  $h$ потенциальная игра с  $L(s)$, а  $g$ с  $Q(s)$, тогда  $u$ потенциальная игра с потенциалом  $L(s) + Q(s)$.

	В нашем случае $h_i(s) = a s_i \implies L(s) = a \sum_{j \in N} s_j$.

	Осталось найти для второго элемента суммы.

	Т.е. хотим 

	 \begin{align*}
		 u_1(s_1, s_2, s_3) - \dots &= P(s_1, s_2, s_3) - \dots \\
		 u_2(s_1, s_2, s_3) - \dots &= P(s_1, s_2, s_3) - \dots \\
		 u_3(s_1, s_2, s_3) - \dots &= P(s_1, s_2, s_3) - \dots
	\end{align*}

	Тогда функция $P$ должна состоять из слагаемых в  $u_1, u_2, u_3$.

	Посмотрим на $g_i(s) = s_i \sum_{j \in N} s_j$, тогда 

	 \begin{align*}
		Q(s) = s_1^2 + s_2^2 + s_3^2 + s_1s_2 + s_1s_3 + s_2s_3
	\end{align*}

	$Q$ мы вычислили как сумму всех разных слагаемых в  $u_1, u_2, u_3$.

	Тогда 

	\begin{align*}
	Q(s_1, s_2, s_3) - Q(s_1', s_2, s_3) &= s_1^2 + s_1s_2 + s_1s_3 - {s_1'}^2 - s_1's_2 - s_1's_3  \\
		&= s_1(s_1 + s_2 + s_3) - s_1'(s_1' + s_2 + s_3) = g_1(s_1, s_2, s_3) - g_1(s_1', s_2, s_3)
	\end{align*}

	В силу симметрии для остальных двух игроков аналогичное утверждение верно.

	Соберем все вместе:

	\begin{align*}
		P(s) = (p - c) \sum_{j \in N} - \sum_{j \in N} s_j^2 - \sum_{\substack{(j, k) \in N^2 \\ j < k}} s_j \cdot s_k
	\end{align*}

\end{exmpl}

\begin{exmpl}

	Пусть Г $= (N, \{s_i\}_{i \in N}, \{u_i\}_{i \in N})$ и

	\begin{align*}
		u_i(s) = A \frac{s_i}{\sum_{j \in N} s_j} - c s_i
	\end{align*}

	Также скажем что $s_i \in (0, A]$.

	Можно представить что $N$ людей подаются на грант, где  $A$ - величина гранта, а  $s_i$ - сколько усилий приложит  $i$ игрок.
	Тогда  $u_i$ - мат. ожидание выигрыша.

	Еще можно сказать что  $i$ игрок покупает  $s_i$ лотерейных билетов и у лотереи один победитель с выигрышем  $A$, тогда  $c$ - цена одного билета. 

	Найдем ординально-потенциальную функцию и покажем что найдется равновесие по Нешу в чистых стратегиях:

	\begin{align*}
		u_i(s) = s_i\left(\frac{A}{\sum_{j \in N} s_j} - c\right)
	\end{align*}

	Предполагаем что выражение под скобками положительно, ведь иначе не выгодно покупать ни одного билета, тогда:

	\begin{align*}
		P(s) &= A - c \sum_{j \in N} s_j \implies \\
		P(s_1, s_2) - P(s_1', s_2) &= (A - c(s_1 + s_2)) - (A - c(s_1' + s_2)) = \\
								   &= -c(s_1 - s_1') \\
		u_1(s_1, s_2) - u_1(s_1', s_2) &= s_1 \left(\frac{A}{s_1 + s_2} - c\right) - s_1'\left(\frac{A}{s_1' + s_2} - c\right) = \\
									   &= \frac{s_1}{s_1 + s_2}(A - c(s_1 + s_2)) - \frac{s_1'}{s_1' + s_2}(A - c(s_1' + s_2))
	\end{align*}

	Можно заметить что мы не угадали с выбором $P$ - это не ординально-потенциальная функция.

	Попробуем еще разок:

	\begin{align*}
		P(s) &= \left(\frac{A}{\sum_{j \in N} s_j} - c\right) \prod_{j \in N} s_j \implies  \\
		P(s_{i}, s_{-i}) - P(s_i', s_{-i}) &= s_i \prod_{j \in N \setminus i } s_j \left(\frac{A}{s_i + \sum_{j \in N \setminus i} s_j} - c\right) - s_i' \prod_{j \in N \setminus i} \left(\frac{A}{s_i' + \sum_{j \in N \setminus i} s_j} - c\right) = \\
										   &= \prod_{j \in N \setminus i} s_j (u_i(s_i, s_{-i} - u_i(s_i' - s_{-i})))
	\end{align*}

	Тогда при $s_j \neq 0, \ \forall j \in N$ получаем что  $u_i(s_i, s_{-i}) - u_i(s_i', s_{-i})$ и  $P(s_i, s_{-i}) - P(s_i', s_{-i})$ одного знака.

	Проверим на $s_i = 0, s_i' \neq 0$, тогда  $u_i(s_i, s_{-i}) = 0$ и  $u_i(s_i', s_{-i}) \neq 0$, но  $P(s_i, s_{-i}) - P(s_i', s_{-i}) = 0$.
\end{exmpl}

Пусть Г - игра в нормальной форме, рассмотрим двух игроков $i, j \in N, \, s_i, s_i' \in S_i, \, s_j, s_j' \in S_j, \, s_{-ij} \in S_{-ij}$, где Г  $= (N, \{s_i\}_{i \in N}, \{u_i\}_{i \in N})$, предположим что Г потенциальная игра с функцией  $P$, тогда:

\begin{align*}
	u_i(s_i, s_j, s_{-ij}) - u_i(s_i', s_j, s_{-ij}) &= P(s_i, s_j, s_{-ij}) - P(s_i', s_j, s_{-ij}) \text{, $j$ меняет стратегию} \\
	u_j(s_i', s_j, s_{-ij}) - u_j(s_i', s_j', s_{-ij}) &= P(s_i', s_j, s_{-ij}) - P(s_i', s_j', s_{-ij}) \text{, $i$ меняет обратно}\\
	u_i(s_i', s_j', s_{-ij}) - u_i(s_i, s_j, s_{-ij}) &= P(s_i', s_j', s_{-ij}) - P(s_i, s_j', s_{-ij}) \text{, $j$ меняет обратно} \\
	u_j(s_i, s_j', s_{-ij}) - u_j(s_i, s_j, s_{-ij}) &= P(s_i, s_j', s_{-ij}) - P(s_i, s_j, s_{-ij})
\end{align*}

Заметим что сумма правых частей равна 0.

А также

\begin{align*}
	u_i(s_i, s_{-i}) - u_i(s_i', s_{-i}) &= P(s_i, s_{-i}) - P(s_i', s_{-i}) \\
	\frac{\partial u_i}{\partial s_i} = \frac{\partial P}{\partial s_i} \quad&\quad \frac{\partial u_j}{\partial s_j} = \frac{\partial P}{\partial s_j} \\
	\frac{\partial^2 u_i}{\partial s_i \partial s_j} = \frac{\partial^2 P}{\partial s_i \partial s_j} \quad&\quad \frac{\partial^2 u_j}{\partial s_j \partial s_i} = \frac{\partial^2 P}{\partial s_j \partial s_i}
\end{align*}


На самом деле каждое из этих условий достаточное:

Теорема Takshi Ui, GFB.

\begin{thm}[Необходимые и достаточные условия существования потенциала] \label{theorem:iff_for_potential}

	Каждое из условий ниже необходимо и достаточно для существования потенциала у игры Г в нормальной форме.

	\begin{enumerate}
		\item \label{theorem_equation:iff_potential_deltas} Для любых  $i, j \in N$ верно  $\Delta_i + \Delta_j + \Delta_i' + \Delta_j' = 0$, где  $k$ по порядку  $\Delta$ это просто выражение в левой части на  $k$ строке выражения сверху \todo[inline]{адекватно написать}.

		\item \label{theorem_equation:iff_potential_partial} $S_i = [a, b], u_i \in C^2$ - дважды дифференцируема по $s_i$ и

			\begin{align*}
				\frac{\partial^2 u_i}{\partial s_i \partial s_j} = \frac{\partial^2 u_j}{\partial s_j \partial s_i} \quad \forall i, j \in N
			\end{align*}

		\item \label{theorem_equation:iff_potential_P_Q}$\exists P: S \to \R, \, \exists Q_i: S_{-i} \to \R$, такие что:

			\begin{align*}
				u_i(s) = P(s) + Q_i(s_{-i})
			\end{align*}

		\item \label{theorem_equation:iff_potential_Phi} $\exists \Phi_K : S_K \to \R, \, K \subseteq N, $, такие что 

			 \begin{align*}
				 u_i(s) = \sum_{\substack{K \subseteq N \\ i \in K} }\Phi_K(s_K)
			\end{align*}

	\end{enumerate}

\end{thm}

\begin{proof}
	Докажем пункт \eqref{theorem_equation:iff_potential_P_Q}:

	Достаточность очевидна.

	В обратную сторону, пусть существует потенциал $P$, тогда:

	\begin{align*}
		u_i(s_i, s_{-i}) - u_i(s_i', s_{-i}) &= P(s_i, s_{-i}) - P(s_i', s_{-i}), \ \forall s_i, s_i' \in S_i, \ \forall s_{-i} \in S_{-i} \\
		u_i(s_i, s_{-i}) - P(s_i, s_{-i}) &= u_i(s_i', s_{-i}) - P(s_i', s_{-i})\\
		R(s) &= u_i(s_i, s_{-i}) - P(s_i, s_{-i}) \implies \\
		R(s_i, s_{-i}) &= R(s_i', s_{-i}), \ \forall s_i, s_i' \in S_i, \ \forall s_{-i} \in S_{-i} \implies \\
		R(s_i, s_{-i}) &= Q(s_{-i}) \implies \\
		u_i(s) &= P(s) + Q_i(s_{-i})
	\end{align*}

	Пункт \eqref{theorem_equation:iff_potential_Phi}:

	Достаточность:

	\begin{align*}
		P(s) = \sum_{\substack{K \subseteq N \\ K \neq \varnothing}} \Phi_K(s_K)
	\end{align*}

	Необходимость:

	Из пункта \eqref{theorem_equation:iff_potential_P_Q} найдутся $P, Q_i$:

	 \begin{align*}
		 u_i(s) = P(s) + Q_i(s_{-i})
	\end{align*}

	Тогда положим 

	\begin{align*}
		\Phi_K(s_K) = \begin{cases}
			P(s) + \sum_{i \in N} Q_i(s_{-i}) &\text{ при } K = N \\
			-Q_i(s_{-i}) &\text{ при } K = N \setminus{i} \\
			0 &\text{ иначе}
		\end{cases}
	\end{align*}

	Тогда:

	\begin{align*}
		\sum_{\substack{K \subseteq N \\ i \in K}} \Phi_K(s_K) &= \Phi_N(s) + \sum_{j \in N \setminus \{i\}} \Phi_{N \setminus \{j\}} (s_{N \setminus \{j\}}) = \\
															   &= P(s) + \sum_{j \in N} Q_j(s_{-j}) - \sum_{j \in N \setminus \{i\}} Q_{j}(s_{-j}) = P(s) + Q_i(s_{-i}) = u_i(s)
	\end{align*}

\end{proof}

\begin{exmpl}[Пример для пункта \eqref{theorem_equation:iff_potential_Phi} Теоремы \ref{theorem:iff_for_potential}]
	Пусть $N = \{1, 2, 3\}$, тогда 

	\begin{align*}
		u_1(s_1, s_2, s_3) &= \Phi_1(s_1) + \Phi_{12}(s_1, s_2) + \Phi_{13}(s_1, s_3) + \Phi_{123}(s_1, s_2, s_3) \\
		u_1(s_1, s_2, s_3) &= \Phi_2(s_2) + \Phi_{12}(s_1, s_2) + \Phi_{23}(s_2, s_3) + \Phi_{123}(s_1, s_2, s_3) \\
		u_1(s_1, s_2, s_3) &= \Phi_3(s_3) + \Phi_{13}(s_1, s_3) + \Phi_{23}(s_2, s_3) + \Phi_{123}(s_1, s_2, s_3)
	\end{align*}

\end{exmpl}

\begin{df}[Модель заполнения]
	Модель заполнения это $(N, M, \{s_i\}{i \in N}, \{c_j\}_{j \in M})$, где  $N \in [n], M \in [m]$.
	$M$ назовем множеством ресурсов, а  $c_j: \R \to \R$ - выплаты от ресурсов.

	 $N$ - кто заполняет(игроки),  $M$ - чем заполняет(ресусы), $s_i$ - стратегии игрока(выбор каких то ресурсов), можно выбрать все ресурсы, а можно частично,  $c_j$ - выплаты от ресурсов. 
\end{df}

\begin{df}[Игра заполнения (congestion game)]
	Игра заполнения это игра игра в нормальной форме $(N, \{s_i\}_{i \in N}, \{u_i\}_{i \in N}$, где 

	\begin{align*}
		u_i(s) &= \sum_{j \in s_i} c_j(k_j(s)), \quad \text{где} \\
		k_j(s) &= \text{кол-во игроков, которые в $s$ выбрали ресурс  $j$}
	\end{align*}

\end{df}

\begin{exmpl}
	Пусть $N = \{1, 2, 3\}, \, M = \{1, 2, 3, 4, 5\}$ и граф ресурсов:
\begin{figure}[ht]
    \centering
	\incfig[0.5]{example-congestion-game}
    \caption{example-congestion-game}
    \label{fig:example-congestion-game}
\end{figure}

Тогда:

\begin{align*}
	u_1(\{1, 2, 3\}, \{2, 3, 4\}, \{3, 4, 5\}) &= c_1(1) + c_2(2) + c_3(3) \\
	u_2(\{1, 2, 3\}, \{2, 3, 4\}, \{3, 4, 5\}) &= c_2(2) + c_3(3) + c_4(2) \\
	u_3(\{1, 2, 3\}, \{2, 3, 4\}, \{3, 4, 5\}) &= c_3(3) + c_4(5) + c_5(1)
\end{align*}

\end{exmpl}

\begin{align*}
	P(s) = \sum_{j \in \cup_{i \in N} s_i} \sum_{k = 1}^{k_j(s)} c_j(k)
\end{align*}

\begin{problem}
	Пусть игроки распределяются на коалиции, а его выигрыш это значение Шепли.

	Пусть $\{\{1, 2, 3\}, \{4, 5\}\}$, тогда считаем что $v$ дана.

	Назовем устойчивым по Нешу, если игроку не выгодно покидать коалицию(и переходить в другие).

	\begin{align*}
		u_i(\pi) &= \sum_{\substack{K \subseteq \pi(i) \\ i \in K}} \frac{\lambda_k(v)}{|k|}, \quad \text{где} \\
		\lambda_k(v) &= \sum_{\substack{R \subseteq K \\ R \neq \varnothing}} (-1)^{|R| - |K|} v(R)
	\end{align*}

	Где $\pi(i)$ - коалиция в которой содержится  $i$ игрок.

	Существует ли устойчивое по Нешу разбиение? Доказать что да
\end{problem}


