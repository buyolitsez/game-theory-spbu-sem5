% 27.10.2023 lecture 01
\documentclass[../main.tex]{subfiles}
\begin{document}

\section{Кооперативная теория игр}

\subsection{Аксиоматика теории Шепли}

\begin{df}[Кооперативная игра]
	Кооперативная игра это пара $(N, v)$, где  $N = \{1, 2, \ldots, n\}$ - множество игроков.

	 $v : 2^{N} \to \R$ - характеристическая функция(показывает суммарный выигрыш коалиции), т.е. $v(\varnothing) = 0$.
\end{df}

\begin{exmpl}
	Пусть есть три музыканта, которые хотят собраться вместе и дать концерт, тогда $N = \{1, 2, 3\}$ и 

	\begin{align*}
		v(\{1\}) = 5 \quad v(\{2\}) &= 7 \quad v(\{3\}) = 4 \\
		v(\{12\}) = 20 \quad v(\{13\}) &= 10 \quad v(\{23\}) = 20 \\
		v(\{123\}) &= 100
	\end{align*}

\end{exmpl}

\begin{df}[Значение кооперативной игры]
	Значением кооперативной игры называется функция $\phi : G(N) \to \R^{n}$, где $G(N)$ - множество характеристических функций с множеством игроков  $N$.
	Т.е. функция $\phi$ задает дележ выигрыша между игроками, если они объединятся в одну большую коалицию.

	Через $\phi_i(v)$ обозначаем выигрыш  $i$ игрока при характеристической функции $v$.
\end{df}

\begin{exmpl}[Парам]
	$L, R, L \cap R = \varnothing, L \cup R = N$ 

	\todo[inline]{ну я тут проморгал :)}
\end{exmpl}

\begin{df}[Игра единодушие (unaminity game)] \label{definition:unaminity_game}
	\begin{align*}
		v_S(K) = \begin{cases}
			1, S \subseteq K \\
			0, \text{иначе}
		\end{cases}
	\end{align*},
	где $S \subseteq N, S \neq \varnothing$.
\end{df}

\begin{exmpl}
	Пусть $N = \{1, 2, 3\}$. Тогда $v_{12}(1) = 0, v_{12}(2) = 0, v_{12}(3) = 0, v_{12}(12) = 1, v_{12}(13) = 0, v_{12}(23) = 0, v_{12}(123) = 1$.
\end{exmpl}

Определим интуитивные аксиомы которым мы хотим чтобы удовлетворял справедливый дележ.

\begin{df}[Аксиомы] \label{axiom:shapley}
	\begin{enumerate}
		\item Аксиома эффективности.\label{axiom:shapley_efficenty} 

		При делении выигрыша не будет остатка, т.е. 

		\begin{align*}
			\sum_{i \in N} \phi_i(v) = v(N)
		\end{align*}
		\item Аксиома нулевого игрока. \label{axiom:shapley_zero_player}

		Если $v(K \cup \{i\}) = v(K), \  \forall K \subseteq N \setminus \{i\}$, то 

		\begin{align*}
			\phi_i(v) = 0
		\end{align*}

	\item Аксиома симметрии. \label{axiom:shapley_symmetry}

		Если $v(K \cup \{i\}) = v(K \cup \{j\}), \ \forall K \subseteq N \setminus \{i, j\}$, то 

		\begin{align*}
			\phi_i(v) = \phi_j(v)
		\end{align*}

	\item Аксиома аддитивности. \label{axiom:shapley_additivity}

		Для любых $\forall v, w \in G(N)$ следует, что

		\begin{align*}
			\phi(v + w) = \phi(v) + \phi(w)
		\end{align*}

		Т.е. неважно на каком этапе делить(как в самом конце, так и например сначала посередине игры, а потом еще раз в конце).
	\end{enumerate}
\end{df}

\begin{thm}[Теорема Шепли] \label{theorem:shepli}
	Аксиомам \ref{axiom:shapley} удовлетворяет единственный дележ:

	\begin{align*}
		\phi_i(v) = \sum_{\substack{K \subseteq N \\ i \in K}} \frac{(|K| - 1)! (|N| - |K|)!}{|N|!} (v(K) - v(K \setminus \{i\}))
	\end{align*}, также называется значением Шепли.

\end{thm}

Доказательство будет приведено дальше.

\begin{lm} \label{lemma:win_unaminity_game}
	Пусть $\phi $ удовлетворяет аксиомам \ref{axiom:shapley}, тогда 

	\begin{align*}
		\phi_i(v_S) = \begin{cases} 
			\frac{1}{|S|}, \quad i \in S \\
			0, \quad \text{иначе}
		\end{cases}
	\end{align*}, где $v_S$ из игры единодушие \ref{definition:unaminity_game}.

\end{lm}

\begin{proof}
	$N \setminus S$ - множество нулевых игроков, по аксиоме \eqref{axiom:shapley_zero_player} следует что  $\phi_i(v_S) = 0, \forall i \in N \setminus S$.

	По аксиоме \eqref{axiom:shapley_efficenty} следует, что  $\sum_{i \in S} \phi_i(v_S) = 1$, но по аксиоме \eqref{axiom:shapley_symmetry} игроки не отличимы, а значит  $\phi_i(v_S) \cdot |S| = 1$.
\end{proof}

\begin{lm} \label{lemma:uniformity_unaminity_game}
	Пусть $\alpha \in R$ - константа, тогда  $\phi_i(\alpha \cdot v_S) = \alpha \phi_i(v_S)$, где $\phi$ удовлетворяет аксиомам \ref{axiom:shapley}.

	Игра опять единодушие \ref{definition:unaminity_game}.
\end{lm}

\begin{proof}
	Тривиально расписывая определение $\phi_i(\alpha \cdot v_S)$ и  $\alpha \cdot \phi_i(v_S)$ по Лемме \ref{lemma:win_unaminity_game} получаем нужное равенство.
\end{proof}

\begin{lm} \label{lemma:basis_to_unaminity_game}
	На множестве $G(N)$ существует базис из  $\{v_S(K)\}$, т.е. для произвольного $v \in G(N)$ выполняется:
	\begin{align*}
		v(K) = \sum_{S \subseteq N} \lambda_S(v) \cdot v_S(K)
	\end{align*}, где 

	\begin{align*}
		\lambda_S(v) = \sum_{R \subseteq S} (-1)^{|S| - |R|} v(R)
	\end{align*}

	Где $v_S$ из игры единодушие \ref{definition:unaminity_game}.

\end{lm}

\begin{exmpl}
	$N = \{1, 2, 3\}$
	\begin{align*}
		v(K) = \lambda_1 v_1(K) + \lambda_2 v_2(K) + \lambda_3 v_3(K) + \lambda_{12} v_{12}(K) + \lambda_{13}v_{13}(K) + \lambda_{23} v_{23}(K) + \lambda_{123} v_{123}(K)
	\end{align*}

	Тогда 

	\begin{align*}
		v(\{1\}) &= \lambda_1, \quad v(\{2\}) = \lambda_2, \quad v(\{3\}) = \lambda_3 \\
		v(\{12\}) &= \lambda_1 + \lambda_2 + \lambda_{12} \implies \lambda_{12} = v(\{12\}) - v(\{1\}) - v(\{2\}) \\
		\lambda_{13} &= v(\{13\}) - v(\{1\}) - v(\{3\}) \\
		\lambda_{23} &= v(\{23\}) - v(\{2\}) - v(\{3\}) \\
		\lambda_{123} &= v(\{123\}) - v(\{12\}) - v(\{13\}) - v(\{23\}) + v(\{1\}) + v(\{2\}) + v(\{3\})
	\end{align*}
\end{exmpl}

\begin{proof}[\normalfont\textsc{Доказательство Леммы \ref{lemma:basis_to_unaminity_game}}]
	Доказательство леммы как в примере применяя индукцию.
\end{proof}

\begin{proof}[\normalfont\textsc{Доказательство Теоремы \ref{theorem:shepli}}]
	Хотим показать, что 

	\begin{align*}
		\phi_i(v) = \sum_{\substack{K \subseteq N \\ i \in K}} \frac{(|K| - 1)! \cdot (|N| - |K|)!}{|N|!} (v(K) - v(K \setminus \{i\}))
	\end{align*}

	Пусть $v \in G(N)$ и пусть  $\phi$ удовлетворяет аксиомам \ref{axiom:shapley}.

	По Лемме \ref{lemma:basis_to_unaminity_game}:
	 \begin{align*}
		&\phi_i(v) = \phi_i \left(\sum_{K \subseteq N} \lambda_K (v) \cdot v_K\right) = & \text{аксиома \eqref{axiom:shapley_additivity}} \\
		&= \sum_{K \subseteq N} \phi_i(\lambda_K(v) \cdot v_K) = & \text{лемма \ref{lemma:uniformity_unaminity_game}}\\
		&= \sum_{K \subseteq N} \lambda_K(v) \cdot \phi_i(v_K) = & \text{лемма \ref{lemma:win_unaminity_game}} \\
		&= \sum_{\substack{K \subseteq N \\ i \in K}} \frac{\lambda_K(v)}{|K|} = &\text{[без подробностей]}\\
		&= \sum_{\substack{K \subseteq N \\ i \in K}} \frac{(|K| - 1)! \cdot (|N| - |K|)!}{|N|} (v(K) - v(K \setminus \{i\}))
	\end{align*}
	Последний переход без подробных объяснений.

	Легко проверить что полученная $\phi$ удовлетворяет аксиомам.

\end{proof}

\begin{exmpl}
	Пусть $N = \{1, 2\}$,  $v \colon 2^{N} \to \R$, тогда

	\begin{align*}
		\phi_1(v) &= \sum_{K \subseteq N, i \in K} \frac{\lambda_K(v)}{|K|} = \lambda_1(v) + \frac{\lambda_{12}(v)}{2} = v(\{1\}) + \frac{v(\{12\}) - v(\{1\}) - v(\{2\})}{2} \\
		\phi_2(v) &=  \sum_{K \subseteq N, i \in K} \frac{\lambda_K(v)}{|K|} = \lambda_2(v) + \frac{\lambda_{12}(v)}{2} = v(\{2\}) + \frac{v(\{12\}) - v(\{1\}) - v(\{2\})}{2}
	\end{align*}
\end{exmpl}

На формулу из Теоремы \ref{theorem:shepli} можно смотреть в терминах перестановок.
Коалиция $i$ игрока это перестановка на  $|N|$ элементах, где  $i$ игрок стоит на  $k$ месте, т.е. до него стоит произвольный  $|K| - 1$ игрок, а после него произвольные $|N| - |K|$ игроков, тогда значение Шепли как бы усредняем выигрыш по всем равновероятным перестановкам.

 \begin{exmpl}
	 Пусть $N = \{1, 2, 3\}$,  $v : 2^{N} \to \R$:

	 \begin{align*}
		 v(\{1\}) = 2 \quad v(\{2\}) &= 3 \quad v(\{3\}) = 2 \\
		 v(\{12\}) = 4 \quad v(\{13\}) &= 2 \quad v(\{23\}) = 5 \\
		 v(\{123\}) &= 10
 	 \end{align*}

	 Выписываем все перестановки:


		{ \centering
\renewcommand{\arraystretch}{1.3}
			\begin{tabular}{l|ccc}
			  & 1 & 2 & 3\\
			  \hline
				1 2 3  & 2 & 2 & 6 \\
				1 3 2 & 2 & 8 & 0 \\
				2 1 3 & 1 & 3 &  6\\
				2 3 1 & 5 & 3 & 2\\
				3 1 2 & 0 & 8 & 2 \\
				3 2 1 & 5 & 3 & 2 \\
				$\phi$ &  $\frac{\sum}{6}$ & $\frac{\sum}{6}$ & $\frac{\sum}{6}$ \\
					   & $\frac{15}{6}$ & $\frac{27}{6}$ & $3$ \\
					   & 2.5 & 4.5 & 3
			\end{tabular}\par
		}

		Пусть перестановка $a, b, c$, тогда считаем выигрыш  $a$, это $v(a)$, выигрыш  $b$ это  $v(ab) - v(a)$, выигрыш  $c$ это  $v(abc) - v(ab)$. 

		И тогда $\phi$ от игрока это сумма в столбце делить на  $|N|!$, т.е. усреднение.

		В нашем примере получили что $\phi_1(v) = 2.5, \phi_2(v) = 4.5, \phi_3(v) = 3$.

		В каком то смысле значение Шепли это мат. ожидание на перестановках, откуда сразу выводится линейность.

	 
\end{exmpl}

\subsection{Потенциальные игры}

\begin{remrk}
	Игра в нормальной форме это $(N, \{S_i\}_{i \in N}, \{u_i\}_{i \in N})$ - множество игроков, стратегий и выигрышей.

	\begin{align*}
		u_1(s_1^*, s_2^*, s_3^*) - u_1(s_1, s_2^*, s_3^*) \geqslant 0, \ \forall s_1 \\
		u_2(s_1^*, s_2^*, s_3^*) - u_2(s_1^*, s_2, s_3^*) \geqslant 0, \ \forall s_2 \\
		u_3(s_1^*, s_2^*, s_3^*) - u_3(s_1^*, s_2^*, s_3) \geqslant 0, \ \forall s_3
	\end{align*} ~--- равновесие по Нешу в случае трёх игроков.

	Существует в смешанных стратегиях, но порой нас интересует в чистых.

	Какие ограничения можно наложить, чтобы такое равновесие в чистых стратегиях всегда существовало?


\end{remrk}


Пусть $P$, такая что 

\begin{align*}
	u_1(s_1^*, s_2^*, s_3^*) - u_1(s_1, s_2^*, s_3^*) = P(s_1^*, s_2^*, s_3^*) - P(s_1, s_2^*, s_3^*), \ \forall s_1 \\
	u_2(s_1^*, s_2^*, s_3^*) - u_2(s_1^*, s_2, s_3^*) = P(s_1^*, s_2^*, s_3^*) - P(s_1^*, s_2, s_3^*), \ \forall s_2 \\
	u_3(s_1^*, s_2^*, s_3^*) - u_3(s_1^*, s_2^*, s_3) = P(s_1^*, s_2^*, s_3^*) - P(s_1^*, s_2^*, s_3), \ \forall s_3 \\
\end{align*}

\begin{df}[Потенциальная игра]
  Игра в нормальной форме $\Gamma = (N, S, u)$ называется потенциальной, если существует $P : S \to \R$, такая что 

	\begin{align*}
		u_i(s_i, s_{-i}) - u_i(s_i', s_{-i}) = P(s_i, s_{-i}) - P(s_i', s_{-i}), \ \forall s_i, s_i' \in S_i, \ \forall s_{-i} \in S_{-i}
	\end{align*}

\end{df}

\begin{claim}
  \label{claim:equlibrim_in_potential_games}
	Пусть $\Gamma$~--- потенциальная игра с потенциалом $P$ и $s^* \in \text{argmax}_{s \in S} P(s)$.
	Тогда  $s^*$~--- равновесие Неша в игре $\Gamma$. 
\end{claim}

\begin{proof}
	\begin{align*}
		P(s^*) - P(s_i, s_{-i}^*) &\geqslant 0 \ \forall s_i \in S_i \implies \\
		u_i(s^*) - u_i(s_i, s^*_{-i}) &\geqslant 0 \ \forall s_i \in S_i
	\end{align*}
\end{proof}

\begin{df}[Ординально-потенциальная игра]
	Игра в нормальной форме Г называется ординально-потенциальной, тогда и только тогда, когда существует $P: \prod_{i \in N} S_i \to \R$ и

	\begin{align*}
		u_i(s_i, s_{-i}) - u_i(s_i', s_{-i}) \geqslant 0 \iff P(s_i, s_{-i}) - P(s_i', s_{-i}) \geqslant 0, \ \forall s_i, s_i' \in S_i, \ \forall s_{-i} \in S_{-i}
	\end{align*}

\end{df}

Аналогично утверждению \ref{claim:equlibrim_in_potential_games} в ординально-потенциальной игре существует равновесие:

\begin{claim}
  $\text{argmax}_{s\in S} P(s)$~--- равновесия Неша в ординально-потенциальной игре.  
\end{claim}

\begin{exmpl}
	Пусть $\Gamma = (N, S, u)$ такая, что $u_i(s) = s_i$, $S_i = [0, A_i]$, тогда  $\Gamma$~--- потенциальная игра с функцией потенциала $P(s) = \sum_{i \in N} s_i$.

	\begin{proof}
		
	  \begin{align*}
		u_i(s_i, s_{-i}) - u_i(s_i', s_{-i}) &= s_i - s_i' = \\
						     &= \left(s_i + \sum_{j \in N \setminus i} s_j\right) - \left(s_i' + \sum_{j \in N \setminus i} s_j\right) = \\ 
						     &= P(s_i, s_{-i}) - P(s_i', s_{-i})
	  \end{align*}
	\end{proof}
\end{exmpl}

\begin{exmpl}[Олигаполия Курно]

	Г $= (N, \{[0, A_i]\}_{i \in N}, \{u_i\}_{i \in N})$, где

	\begin{align*}
		u_i(s) = (p - \sum_{i \in N} s_i) \cdot s_i - c_i s_i, \quad s_i \in [0, A_i]
	\end{align*}

	Идейно, фирма решает сколько товара ей надо производить(обратная функция спроса).
	Чем больше товара мы выпустили, тем цена меньше и есть еще затраты на производство.

	Считаем, что $p - n A \geqslant 0$(т.е. цена положительна сколько бы товара мы не выпускали).

	Неравенство Неша существует, т.к. функции вогнутые, а пространство компактно.

\end{exmpl}

\begin{problem}
	Найти потенциальную функцию в примере выше.
\end{problem}

\begin{problem}
	Пусть $N = \{1, 2, 3\}$ и 

	\begin{align*}
		v(1) = 5 \quad v(2) &= 6 \quad v_3 = 4 \\
		v(12) = 8 \quad v(13) &= 6 \quad v(23) = 10 \\
		v(123) = 15
	\end{align*}

	Вычислить значение Шепли:  $\phi_1, \phi_2, \phi_3$ - ?

\end{problem}


\end{document}
